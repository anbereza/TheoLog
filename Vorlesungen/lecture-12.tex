\documentclass[aspectratio=1610,onlymath]{beamer}
% \documentclass[aspectratio=1610,onlymath,handout]{beamer}

\input{macros-lecture}

\defineTitle{12}{\PSpace-Vollständigkeit}{3. Juni 2021}

\begin{document}

\maketitle

\begin{frame}\frametitle{\complclass{NLogSpace}}

\NLogSpace: Probleme, die eine nichtdeterministische TM mit logarithmischem Speicher lösen kann


\begin{itemize}
\item Deterministisch lösbar in polynomieller Zeit
\item In der Regel deutlich besser parallelisierbar als typische Probleme in \PTime
\item Aber: seriell nicht unbedingt schneller lösbar als andere Probleme in \PTime
\end{itemize}
\bigskip

\emph{Typisches Beispiel:}
\Slang{Erreichbarkeit} in gerichteten Graphen


\end{frame}

\begin{frame}\frametitle{\PSpace}

\PSpace-vollständige Probleme\\ = Probleme, die mindestens so schwer
sind, wie alle anderen Probleme in \PSpace\\
= die schwersten Probleme in \PSpace.
\bigskip

\emph{Vermutung:}\\
Kein \PSpace-vollständiges Problem ist in NP,\\ d.h. \PSpace
ist echt schwerer als NP und \Scomplclass{coNP}
\bigskip

\emph{\PSpace-vollständige Probleme}
\begin{itemize}
\item \Slang{TrueQBF:} Wahrheit von quantifizierten Booleschen Formeln (und das Formelspiel)
\item \Slang{Geography:} Spiel auf einem Graphen
\end{itemize}


\end{frame}

\sectionSlide{\PSpace-Vollständigkeit}

\begin{frame}\frametitle{Rückblick: \Slang{TrueQBF} in \PSpace}

\theobox{\emph{Satz:} \Slang{TrueQBF} ist in \PSpace.}\pause

\emph{Beweis:} Durch Angabe eines (Pseudo-)Algorithmus

\codebox{
{\tt%
\alert{01}~\textsc{TrueQBF}($F$) : \\
\alert{02}~~~\Scode{if} $F$ "`hat keine Quantoren"' :\\
\alert{03}~~~~~~\Scode{return} "`Aussagenlogische Auswertung von $F$"'\\
\alert{04}~~~\Scode{else if} $F = \exists p.G$ :\\
\alert{05}~~~~~~\Scode{return} (\textsc{TrueQBF}($G[p/\top]$) \Scode{OR} \textsc{TrueQBF}($G[p/\bot]$))\\
\alert{06}~~~\Scode{else if} $F = \forall p.G$ :\\
\alert{07}~~~~~~\Scode{return} (\textsc{TrueQBF}($G[p/\top]$) \Scode{AND} \textsc{TrueQBF}($G[p/\bot]$))
}
}\smallskip

\begin{itemize}
\item Evaluation in Zeile \texttt{03} möglich in \PSpace
\item Rekursion in Zeilen \texttt{05} und \texttt{07} können der Reihe nach abgearbeitet werden, wobei Speicher wiederverwendet wird
\item Jeder Rekursionsschritt benötigt polynomiellen Speicher
\item Maximale Rekursionstiefe = Zahl der Atome (linear)\qed
\end{itemize}
% $\leadsto$ \PSpace-Algorithmus


\end{frame}

\begin{frame}\frametitle{\PSpace-Schwere}

\defbox{Ein Problem \Slang{Q} ist \redalert{\PSpace-schwer}, wenn
für jedes Problem \Slang{P} in \PSpace ein polynomielle Reduktion $\Slang{P}\leq_p \Slang{Q}$ existiert. \Slang{Q} ist \redalert{\PSpace-vollständig}, wenn es \PSpace-schwer ist und in \PSpace liegt.}\bigskip\pause

\theobox{\emph{Satz:} \Slang{TrueQBF} ist \PSpace-schwer.}\pause

\emph{Beweisidee:}
\begin{itemize}
\item Wie beim Satz von Cook/Levin stellen wir den Lauf einer Turingmaschine
mit aussagenlogischen Atomen dar
\item Speicher ist wie bei NP polynomiell (einfach kodierbar)
\item Zeit ist problematisch: in \PSpace kann ein TM exponentiell lang laufen
$\leadsto$ wir können nicht einfach für jeden Zeitschritt eine neue Version
der Konfigurationsvariablen anlegen und jeden Übergang mit Formeln axiomatisieren
\end{itemize}

\end{frame}

\begin{frame}\frametitle{Beweisidee (Fortsetzung)}

\emph{Einsichten:}
\begin{itemize}
\item Man kann eine komplette TM-Konfiguration in polynomiell vielen Atomen
  kodieren (wie bei Cook-Levin)
\item Ebenso kann man zwei oder drei Konfigurationen kodieren -- aber nicht exponentiell viele (für jeden Schritt)
\item Direkte Übergänge, Start- und Endkonfiguration kann man mit Formeln axiomatisieren
\end{itemize}\bigskip\pause

\emph{Akzeptanz ausdrücken}
\begin{itemize}
\item Gewünschte Aussage: "`Ausgehend von der Startkonfiguration kann eine Endkonfiguration in endlich vielen Übergängen erreicht werden"'
\item = Erreichbarkeitsproblem in einem exponentiell großen Graphen
\end{itemize}


\end{frame}

\begin{frame}\frametitle{Beweisidee (Fortsetzung)}

\alert{Erreichbarkeitsproblem in einem exponentiell großen Graphen}
\bigskip\pause

\emph{Lösung:} "`Middle-First-Suche"'
\begin{itemize}
\item Um zu prüfen, ob man von $s$ nach $t$ gelangen kann
\item Prüfe zunächst, ob $s=t$ oder $s$ direkter Vorgänger von $t$
\item Rate einen Punkt $m$ in der "`Mitte"' eines Pfades von $s$ nach $t$
\item Prüfe (rekursiv) ob man von $s$ nach $m$ gelangen kann
\item Prüfe (rekursiv) ob man von $m$ nach $t$ gelangen kann
\end{itemize}\pause
$\leadsto$ Anzahl der zu ratenden Mittelpunkte ist logarithmisch in Länge des Pfades\\
$\leadsto$ Speicher kann in jedem Schritt wiederverwendet werden
\bigskip

Diesen Algorithmus kann man polynomiell in QBF kodieren

\end{frame}

\begin{frame}\frametitle{Alternierende Quantoren}

% Wir sehen leicht, dass \PSpace-Vollständigkeit auch bei einem Sonderfall von
% \Slang{TrueQBF} gilt:

\defbox{\Slang{TrueQBF}${}_{\Slang{alt}}$ ist das Problem \Slang{TrueQBF},
beschränkt auf QBF der Form $\exists p_1.\forall p_2.\exists p_3.\ldots\forall p_{\ell-1}.\exists p_\ell. F$, wobei $F$ in Klauselform ist.}

Diese Version entspricht eher der Idee eines Spiels, bei dem Emilia und Anton abwechselnd
Belegungen festlegen. Emilia erhält den ersten und den letzten Zug.\bigskip\pause

\theobox{\emph{Satz:} \Slang{TrueQBF}${}_{\Slang{alt}}$ ist \PSpace-vollständig.}

\pause\emph{Beweis:} $\Slang{TrueQBF}_{\Slang{alt}}\in\PSpace$ folgt aus $\Slang{TrueQBF}\in\PSpace$.\bigskip

Für die Schwere zeigen wir $\Slang{TrueQBF}\leq_p\Slang{TrueQBF}_{\Slang{alt}}$:
\begin{itemize}
\item man fügt einfach nach Bedarf zusätzliche Quantoren ein
\item es müssen nicht alle quantifizierten Atome in $F$ vorkommen\qed
\end{itemize}

\end{frame}

\begin{frame}\frametitle{Rückblick: Geography}

\alert{Ein Kinderspiel:}
\begin{itemize}
\item Zwei Spieler benennen abwechselnd Städte
\item Jede Stadt muss mit dem letzten Buchstaben der zuvor genannten beginnen
\item Wiederholungen sind verboten
\item Der erste Spieler, der keine Stadt mehr nennen kann, verliert
\end{itemize}
\smallskip

\alert{Ein Mathematikerspiel:}
\begin{itemize}
\item Zwei Spieler markieren Knoten in einem gerichteten Graph
\item Jeder Knoten muss ein Nachfolger des vorigen sein
\item Wiederholungen sind verboten
\item Der erste Spieler, der keinen Knoten markieren kann, verliert
\end{itemize}
\smallskip

% \pause
\defbox{Entscheidungsproblem \Slang{Geography}:\\[1ex]
\emph{Gegeben:} Ein gerichteter Graph und ein Startknoten\\
\emph{Frage:} Hat Emilia eine Gewinnstrategie für dieses Spiel?
}

\end{frame}


\begin{frame}[t]\frametitle{\Slang{Geography} ist \PSpace-vollständig \ghost{(1)}}

\vspace{-1ex}
\theobox{\emph{Satz:} \Slang{Geography} ist \PSpace-vollständig.}\pause

\emph{Beweis:} (1) \Slang{Geography} ist in \PSpace\pause
\smallskip

Bei Startknoten $s$ und Graph $G$ erhalten wir die Antwort durch Aufruf von
{\tt \textsc{eCanWin}($G$, $s$, $\{s\}$)}, definiert wie folgt:

\codebox{
{\tt%
\alert{01}~\textsc{eCanWin}($G$, $n$, Visited) : \\
\alert{02}~~~result = \Scodelit{false}\\
\alert{03}~~~\Scode{for all} Nachfolger${}^*$ $m$ von $n$ in $G$ mit $m\notin{}$Visited :\\
\alert{04}~~~~~~result = result \Scode{OR} aCannotWin($G$,$m$,Visited${}\cup\{m\}$)\\
\alert{05}~~~\Scode{return} result\\
~\\
\alert{06}~\textsc{aCannotWin}($G$, $n$, Visited) : \\
\alert{07}~~~result = \Scodelit{true}\\
\alert{08}~~~\Scode{for all} Nachfolger${}^*$ $m$ von $n$ in $G$ mit $m\notin{}$Visited :\\
\alert{09}~~~~~~result = result \Scode{AND} eCanWin($G$,$m$,Visited${}\cup\{m\}$)\\
\alert{10}~~~\Scode{return} result
}
}

\ghost{\hspace{13.8cm}\rotatebox{90}{~\hspace{5mm}\footnotesize${}^*$ gemeint sind jeweils direkte Nachfolger}}
\end{frame}


\begin{frame}[t]\frametitle{\Slang{Geography} ist \PSpace-vollständig \ghost{(2)}}

\vspace{-1ex}
\theobox{\emph{Satz:} \Slang{Geography} ist \PSpace-vollständig.}\pause

\emph{Beweis:} (2) \Slang{Geography} ist \PSpace-schwer\pause
\smallskip

Durch Reduktion $\Slang{TrueQBF}_{\Slang{alt}}\leq_p \Slang{Geography}$:

Wir betrachten eine QBF $\exists p_1.\forall p_2\ldots \exists p_\ell. F$
und konstruieren daraus einen Graphen für \Slang{Geography} \bigskip

\alert{Grundidee:} {\tiny (siehe auch M. Sipser: Introduction to the Theory of Computation; 2013; Thm. 8.14)}
\begin{itemize}
\item Der Graph beginnt mit einer Abfolge von Rauten-Strukturen, welche die 
Entscheidungen der Spieler darstellen: für jedes Atom $p$ gibt es dort zwei Knoten, $p_{\mytrue}$ und $p_{\myfalse}$, von denen genau einer besucht wird
\item Es folgt eine Baumstruktur: ein Knoten für $F$, darunter ein Knoten für jede Klausel und darunter jeweils ein Knoten für jedes Literal
\item Von jedem Literal $p$ bzw. $\neg p$ gibt es eine Kante zurück zum Knoten $p_{\mytrue}$ bzw. $p_{\myfalse}$
\end{itemize}


\end{frame}

\begin{frame}<beamer: 0>[t, label=gameframe]\frametitle{\Slang{Geography} ist \PSpace-vollständig: Beispiel}

Wir betrachten die Formel $\exists r.\forall p.\exists q.(r\vee q\vee p)\wedge(\neg p\vee q)\wedge(\neg q\vee p)$


\begin{center}
\scalebox{0.7}{%
\begin{tikzpicture}[scale=1.1]
% \draw[help lines] (0,0) grid (7,2);
\node (rs) [rectangle,rounded corners=1.5ex, minimum width=1.5em, minimum height=1.5em, draw=black,thick,onslide={<2->{fill=lightgray}}] at (0,0) {$r_s$};
\node (r0) [rectangle,rounded corners=1.5ex, minimum width=1.5em, minimum height=1.5em,draw=black,thick] at (-1,-1) {$r_{\myfalse}$};
\node (r1) [rectangle,rounded corners=1.5ex, minimum width=1.5em, minimum height=1.5em, draw=black,thick,onslide={<3->{fill=strongyellow}}] at (1,-1) {$r_{\mytrue}$};
\node (re) [rectangle,rounded corners=1.5ex, minimum width=1.5em, minimum height=1.5em, draw=black,thick,onslide={<4->{fill=lightpurple}}] at (0,-2) {$r_e$};

\visible<2->{
\node [rectangle,rounded corners=1.5ex, minimum width=1.5cm, minimum height=1.5em, draw=black,thick,fill=lightgray] at (9,-0.5) {Start};
\node [rectangle,rounded corners=1.5ex, minimum width=1.5cm, minimum height=1.5em, draw=black,thick,fill=strongyellow] at (9,-1.1) {Emilia};
\node [rectangle,rounded corners=1.5ex, minimum width=1.5cm, minimum height=1.5em, draw=black,thick,fill=lightpurple] at (9,-1.7) {Anton};
}

\path[->,line width=0.5mm](-1,0) edge (rs);

\path[->,line width=0.5mm](rs) edge (r0);
\path[->,line width=0.5mm](rs) edge (r1);
\path[->,line width=0.5mm](r0) edge (re);
\path[->,line width=0.5mm](r1) edge (re);

\node (ps) [rectangle,rounded corners=1.5ex, minimum width=1.5em, minimum height=1.5em, draw=black,thick,onslide={<5->{fill=strongyellow}}] at (0,-3) {$p_s$};
\node (p0) [rectangle,rounded corners=1.5ex, minimum width=1.5em, minimum height=1.5em, draw=black,thick,onslide={<6->{fill=lightpurple}}] at (-1,-4) {$p_{\myfalse}$};
\node (p1) [rectangle,rounded corners=1.5ex, minimum width=1.5em, minimum height=1.5em, draw=black,thick] at (1,-4) {$p_{\mytrue}$};
\node (pe) [rectangle,rounded corners=1.5ex, minimum width=1.5em, minimum height=1.5em, draw=black,thick,onslide={<7->{fill=strongyellow}}] at (0,-5) {$p_e$};

\path[->,line width=0.5mm](re) edge (ps);

\path[->,line width=0.5mm](ps) edge (p0);
\path[->,line width=0.5mm](ps) edge (p1);
\path[->,line width=0.5mm](p0) edge (pe);
\path[->,line width=0.5mm](p1) edge (pe);

\node (qs) [rectangle,rounded corners=1.5ex, minimum width=1.5em, minimum height=1.5em, draw=black,thick,onslide={<8->{fill=lightpurple}}] at (0,-6) {$q_s$};
\node (q0) [rectangle,rounded corners=1.5ex, minimum width=1.5em, minimum height=1.5em, draw=black,thick,onslide={<9->{fill=strongyellow}}] at (-1,-7) {$q_{\myfalse}$};
\node (q1) [rectangle,rounded corners=1.5ex, minimum width=1.5em, minimum height=1.5em, draw=black,thick] at (1,-7) {$q_{\mytrue}$};
\node (qe) [rectangle,rounded corners=1.5ex, minimum width=1.5em, minimum height=1.5em, draw=black,thick,onslide={<10->{fill=lightpurple}}] at (0,-8) {$q_e$};

\path[->,line width=0.5mm](pe) edge (qs);

\path[->,line width=0.5mm](qs) edge (q0);
\path[->,line width=0.5mm](qs) edge (q1);
\path[->,line width=0.5mm](q0) edge (qe);
\path[->,line width=0.5mm](q1) edge (qe);

\node (Q) [rectangle,rounded corners=1.5ex, minimum width=1.5em, minimum height=1.5em, draw=black,thick,onslide={<11->{fill=strongyellow}}] at (9,-7.5) {$Q$};

\node (rpq) [rectangle,rounded corners=1.5ex, minimum width=1.5em, minimum height=1.5em, draw=black,thick] at (6,-1) {$r\vee q \vee p$};

\node (npq) [rectangle,rounded corners=1.5ex, minimum width=1.5em, minimum height=1.5em, draw=black,thick,onslide={<12->{fill=lightpurple}}] at (6,-4) {$\neg p\vee q$};

\node (nqp) [rectangle,rounded corners=1.5ex, minimum width=1.5em, minimum height=1.5em, draw=black,thick] at (6,-7) {$\neg q\vee p$};

\path[->,line width=0.5mm,bend right=10](Q) edge (rpq);
\path[->,line width=0.5mm,bend right=15](Q) edge (npq);
\path[->,line width=0.5mm,bend right=5](Q) edge (nqp);

\node (litr) [rectangle,rounded corners=1.5ex, minimum width=1.5em, minimum height=1.5em, draw=black,thick] at (3,0) {$r$};
\node (litp) [rectangle,rounded corners=1.5ex, minimum width=1.5em, minimum height=1.5em, draw=black,thick] at (3,-2) {$p$};
\node (litnp) [rectangle,rounded corners=1.5ex, minimum width=1.5em, minimum height=1.5em, draw=black,thick,onslide={<13->{fill=strongyellow}}] at (3,-4) {$\neg p$};
\node (litq) [rectangle,rounded corners=1.5ex, minimum width=1.5em, minimum height=1.5em, draw=black,thick] at (3,-6) {$q$};
\node (litnq) [rectangle,rounded corners=1.5ex, minimum width=1.5em, minimum height=1.5em, draw=black,thick] at (3,-8) {$\neg q$};

\path[->,line width=0.5mm](rpq) edge (litr);
\path[->,line width=0.5mm](rpq) edge (litp);
\path[->,line width=0.5mm,bend right=10](rpq) edge (litq);

	\path[->,line width=1.1mm,white](npq) edge (litnp);
\path[->,line width=0.5mm](npq) edge (litnp);
\path[->,line width=0.5mm](npq) edge (litq);

\path[->,line width=0.5mm](nqp) edge (litnq);
	\path[->,line width=1.1mm,bend right=10,white](nqp) edge (litp);
\path[->,line width=0.5mm,bend right=10](nqp) edge (litp);

\path[->,line width=0.5mm,bend right=5](litr) edge (r1);
\path[->,line width=0.5mm,bend right=10](litp) edge (p1);
\path[->,line width=0.5mm,bend right=10](litq) edge (q1);
	\path[->,line width=1.1mm,bend left=20,white](litnp) edge (p0);
\path[->,line width=0.5mm,bend left=20](litnp) edge (p0);
	\path[->,line width=1.1mm,bend left=5,white](litnq) edge (q0);
\path[->,line width=0.5mm,bend left=5](litnq) edge (q0);

\path[->,line width=0.5mm,bend right=15](qe) edge (Q);
\end{tikzpicture}}
\end{center}

\end{frame}


\againframe<1>{gameframe}

\begin{frame}[t]\frametitle{\Slang{Geography} ist \PSpace-vollständig \ghost{(3)}}

\vspace{-1ex}
\theobox{\emph{Satz:} \Slang{Geography} ist \PSpace-vollständig.}

\emph{Beweis:} (2) \Slang{Geography} ist \PSpace-schwer
\smallskip

Beweisidee:
\begin{itemize}
\item Zuerst bestimmen Anton und Emilia abwechselnd Wahrheitswerte, indem sie die Rauten
jeweils links oder rechts durchlaufen
\item Danach darf Anton zur Auswertung eine Klausel auswählen (die er für nicht erfüllt hält)
\item Anschließend wählt Emilia ein Literal dieser Klausel (welches ihrer Meinung nach erfüllt ist)
\item Emilia gewinnt, wenn das Literal wirklich erfüllt war (denn dann gibt es keinen Weg zu einem noch nicht besuchten Knoten)
\item Anton gewinnt, wenn das Literal nicht erfüllt war (denn dann kann er den Weg noch genau einen Schritt fortsetzen)
\end{itemize}

\end{frame}

\againframe{gameframe}

% \begin{frame}[t, label=game_frame]\frametitle{\Slang{Geography} ist \PSpace-vollständig: Beispiel}
% 
% Wir betrachten die Formel $\exists r.\forall p.\exists q.(r\vee q\vee p)\wedge(\neg p\vee q)\wedge(\neg q\vee p)$
% 
% 
% \begin{center}
% \scalebox{0.7}{%
% \begin{tikzpicture}[scale=1.1]
% % \draw[help lines] (0,0) grid (7,2);
% \node (rs) [rectangle,rounded corners=1.5ex, minimum width=1.5em, minimum height=1.5em, draw=black,thick,onslide={<2->{fill=lightgray}}] at (0,0) {$r_s$};
% \node (r0) [rectangle,rounded corners=1.5ex, minimum width=1.5em, minimum height=1.5em,draw=black,thick] at (-1,-1) {$r_{\myfalse}$};
% \node (r1) [rectangle,rounded corners=1.5ex, minimum width=1.5em, minimum height=1.5em, draw=black,thick,onslide={<3->{fill=strongyellow}}] at (1,-1) {$r_{\mytrue}$};
% \node (re) [rectangle,rounded corners=1.5ex, minimum width=1.5em, minimum height=1.5em, draw=black,thick,onslide={<4->{fill=lightpurple}}] at (0,-2) {$r_e$};
% 
% \visible<2->{
% \node [rectangle,rounded corners=1.5ex, minimum width=1.5cm, minimum height=1.5em, draw=black,thick,fill=lightgray] at (9,-0.5) {Start};
% \node [rectangle,rounded corners=1.5ex, minimum width=1.5cm, minimum height=1.5em, draw=black,thick,fill=strongyellow] at (9,-1.1) {Emilia};
% \node [rectangle,rounded corners=1.5ex, minimum width=1.5cm, minimum height=1.5em, draw=black,thick,fill=lightpurple] at (9,-1.7) {Anton};
% }
% 
% \path[->,line width=0.5mm](-1,0) edge (rs);
% 
% \path[->,line width=0.5mm](rs) edge (r0);
% \path[->,line width=0.5mm](rs) edge (r1);
% \path[->,line width=0.5mm](r0) edge (re);
% \path[->,line width=0.5mm](r1) edge (re);
% 
% \node (ps) [rectangle,rounded corners=1.5ex, minimum width=1.5em, minimum height=1.5em, draw=black,thick,onslide={<5->{fill=strongyellow}}] at (0,-3) {$p_s$};
% \node (p0) [rectangle,rounded corners=1.5ex, minimum width=1.5em, minimum height=1.5em, draw=black,thick,onslide={<6->{fill=lightpurple}}] at (-1,-4) {$p_{\myfalse}$};
% \node (p1) [rectangle,rounded corners=1.5ex, minimum width=1.5em, minimum height=1.5em, draw=black,thick] at (1,-4) {$p_{\mytrue}$};
% \node (pe) [rectangle,rounded corners=1.5ex, minimum width=1.5em, minimum height=1.5em, draw=black,thick,onslide={<7->{fill=strongyellow}}] at (0,-5) {$p_e$};
% 
% \path[->,line width=0.5mm](re) edge (ps);
% 
% \path[->,line width=0.5mm](ps) edge (p0);
% \path[->,line width=0.5mm](ps) edge (p1);
% \path[->,line width=0.5mm](p0) edge (pe);
% \path[->,line width=0.5mm](p1) edge (pe);
% 
% \node (qs) [rectangle,rounded corners=1.5ex, minimum width=1.5em, minimum height=1.5em, draw=black,thick,onslide={<8->{fill=lightpurple}}] at (0,-6) {$q_s$};
% \node (q0) [rectangle,rounded corners=1.5ex, minimum width=1.5em, minimum height=1.5em, draw=black,thick,onslide={<9->{fill=strongyellow}}] at (-1,-7) {$q_{\myfalse}$};
% \node (q1) [rectangle,rounded corners=1.5ex, minimum width=1.5em, minimum height=1.5em, draw=black,thick] at (1,-7) {$q_{\mytrue}$};
% \node (qe) [rectangle,rounded corners=1.5ex, minimum width=1.5em, minimum height=1.5em, draw=black,thick,onslide={<10->{fill=lightpurple}}] at (0,-8) {$q_e$};
% 
% \path[->,line width=0.5mm](pe) edge (qs);
% 
% \path[->,line width=0.5mm](qs) edge (q0);
% \path[->,line width=0.5mm](qs) edge (q1);
% \path[->,line width=0.5mm](q0) edge (qe);
% \path[->,line width=0.5mm](q1) edge (qe);
% 
% \node (Q) [rectangle,rounded corners=1.5ex, minimum width=1.5em, minimum height=1.5em, draw=black,thick,onslide={<11->{fill=strongyellow}}] at (9,-7.5) {$Q$};
% 
% \node (rpq) [rectangle,rounded corners=1.5ex, minimum width=1.5em, minimum height=1.5em, draw=black,thick] at (6,-1) {$r\vee q \vee p$};
% 
% \node (npq) [rectangle,rounded corners=1.5ex, minimum width=1.5em, minimum height=1.5em, draw=black,thick,onslide={<12->{fill=lightpurple}}] at (6,-4) {$\neg p\vee q$};
% 
% \node (nqp) [rectangle,rounded corners=1.5ex, minimum width=1.5em, minimum height=1.5em, draw=black,thick] at (6,-7) {$\neg q\vee p$};
% 
% \path[->,line width=0.5mm,bend right=10](Q) edge (rpq);
% \path[->,line width=0.5mm,bend right=15](Q) edge (npq);
% \path[->,line width=0.5mm,bend right=5](Q) edge (nqp);
% 
% \node (litr) [rectangle,rounded corners=1.5ex, minimum width=1.5em, minimum height=1.5em, draw=black,thick] at (3,0) {$r$};
% \node (litp) [rectangle,rounded corners=1.5ex, minimum width=1.5em, minimum height=1.5em, draw=black,thick] at (3,-2) {$p$};
% \node (litnp) [rectangle,rounded corners=1.5ex, minimum width=1.5em, minimum height=1.5em, draw=black,thick,onslide={<13->{fill=strongyellow}}] at (3,-4) {$\neg p$};
% \node (litq) [rectangle,rounded corners=1.5ex, minimum width=1.5em, minimum height=1.5em, draw=black,thick] at (3,-6) {$q$};
% \node (litnq) [rectangle,rounded corners=1.5ex, minimum width=1.5em, minimum height=1.5em, draw=black,thick] at (3,-8) {$\neg q$};
% 
% \path[->,line width=0.5mm](rpq) edge (litr);
% \path[->,line width=0.5mm](rpq) edge (litp);
% \path[->,line width=0.5mm,bend right=10](rpq) edge (litq);
% 
% 	\path[->,line width=1.1mm,white](npq) edge (litnp);
% \path[->,line width=0.5mm](npq) edge (litnp);
% \path[->,line width=0.5mm](npq) edge (litq);
% 
% \path[->,line width=0.5mm](nqp) edge (litnq);
% 	\path[->,line width=1.1mm,bend right=10,white](nqp) edge (litp);
% \path[->,line width=0.5mm,bend right=10](nqp) edge (litp);
% 
% \path[->,line width=0.5mm,bend right=5](litr) edge (r1);
% \path[->,line width=0.5mm,bend right=10](litp) edge (p1);
% \path[->,line width=0.5mm,bend right=10](litq) edge (q1);
% 	\path[->,line width=1.1mm,bend left=20,white](litnp) edge (p0);
% \path[->,line width=0.5mm,bend left=20](litnp) edge (p0);
% 	\path[->,line width=1.1mm,bend left=5,white](litnq) edge (q0);
% \path[->,line width=0.5mm,bend left=5](litnq) edge (q0);
% 
% \path[->,line width=0.5mm,bend right=15](qe) edge (Q);
% \end{tikzpicture}}
% \end{center}
% 
% \end{frame}

\sectionSlide{Weitere Komplexitätsklassen}

\begin{frame}\frametitle{Komplexität jenseits von \PSpace?}

Wir haben uns auf drei Klassen konzentriert:
\[ \PTime \subseteq \NP \subseteq \PSpace\]

Ersetzt man "`polynomiell"' durch "`exponentiell"', so erhält man jenseits von $\PSpace$ ein ganz ähnliches Bild:
\[ \ExpTime \subseteq \NExpTime \subseteq \ExpSpace\]\pause

Man kann beliebig hohe Türme von Exponenten konstruieren:
\[ \TwoExpTime \subseteq \NTwoExpTime \subseteq \complclass{2ExpSpace}\]
\[ \complclass{3ExpTime} \subseteq \complclass{N3ExpTime} \subseteq \complclass{3ExpSpace}\]
\[\vdots\]

Vieles, was wir gelernt haben, gilt hier wie zuvor -- aber diese Klassen sind von immer
geringerer praktischer Relevanz

\end{frame}

\begin{frame}\frametitle{Gibt es so schwere Probleme?}\pause

Für $k$-$\complclass{Exp}$-Klassen lassen sich künstliche Probleme leicht finden: %\\
"`Gegeben eine det.\ Turingmaschine $\mathcal{M}$ und eine Eingabe $w$, wird $\mathcal{M}$ das Wort $w$ in Zeit $\underbrace{2^{.^{.^{.^{2^{|w|}}}}}}_{k\text{-mal 2}}$ akzeptieren?"'\\[-1ex]
{\tiny (Warum ist das $k$-$\complclass{Exp}$-vollständig? Funktioniert die Reduktion z.B. von Problemen, die in Zeit $2^{.^{.^{.^{2^{|w|^{42}}}}}}\!\!\!\!$ lösbar sind?)}
\bigskip\pause

Jenseits von \ExpSpace gibt es nur wenige relevante Probleme.\medskip

\examplebox{\emph{Beispiel:} Der W3C-Standard \alert{Web Ontology Language} (OWL, Version 2)
definiert die logische Beschreibungssprache OWL~2~DL. Logisches Schließen in dieser Sprache ist \NTwoExpTime-vollständig.
}\pause

Es gibt aber auch entscheidbare Probleme, die durch keine vielfach exponentiell 
beschränkte TM entschieden werden: Dies sind Probleme \redalert{nichtelementarer}
Komplexität\medskip

\examplebox{\emph{Beispiel:} Die Äquivalenz von regulären Ausdrücken mit einem zusätzlichen
Komplementierungsoperator ist entscheidbar, aber nicht elementar.
}\pause

Viele schwere praktische Probleme sind einfach unentscheidbar\\[-1ex]
{\tiny (z.B. Syntax von C++)}

\end{frame}

\begin{frame}\frametitle{Komplexitäten unterhalb von \PTime?}

Wir haben die folgenden Komplexitäten betrachtet
\[\LogSpace \subseteq \NLogSpace \subseteq \PTime\]

Hier gibt es bereits \emph{technische Probleme:}
\begin{itemize}
\item TMs mit logarithmischem Speicher benötigen getrennte Ein- und Ausgabebänder
\item Logarithmen sind nicht abgeschlossen unter polynomiellen Operationen;
z.B. sagt uns Savitch \alert{nicht} ob $\LogSpace \stackrel{?}{=} \NLogSpace$
\end{itemize}\pause\smallskip

Will man \emph{noch kleinere Komplexitätsklassen} betrachten, dann muss man zu anderen
Berechnungsmodellen greifen:
\begin{itemize}
\item Schaltkreise statt TMs ($\leadsto$ \redalert{circuit complexity})
\item Logarithmisch zeitbeschränkte TMs mit beschränkter Parallelverarbeitung ($\leadsto$ \redalert{alternating, random access TM})
\item Automatenmodelle ($\leadsto$ DFA, PDA, \ldots)
\end{itemize}
Manche der Klassen sind dann sehr stark von Details der Problemdefinition
abhängig (wenig robust)

\end{frame}

\begin{frame}\frametitle{Gibt es so leichte Probleme?}

Auch bei den subpolynomiellen Klassen ergeben sich in der Regel typische Probleme
aus der Definition.
\bigskip\pause

Es gibt eine Reihe \alert{interessanter, praktisch relevanter \LogSpace-Probleme}, z.B.
\begin{itemize}
\item Erreichbarkeit in ungerichteten Graphen (Omar Reingold, 2005)
\item Zwei-Färbbarkeit von Graphen
\end{itemize}
\bigskip\pause

Es gibt auch \alert{noch einfachere praktisch relevante Probleme}, z.B.
\begin{itemize}
\item logische und arithmetische Operationen
\item Beantwortung von fest vorgegebenen SQL-Anfragen auf beliebigen Datenbanken
\item Wortprobleme bei endlichen Automaten
\end{itemize}

In diesen Fällen wird es immer schwerer, von "`Schwere"' und "`Vollständigkeit"' zu sprechen

\end{frame}

\begin{frame}\frametitle{Komplexität und Spiele}

\alert{\NP} ist eine typische Klasse für Solitaire-Spiele:
\begin{itemize}
\item Sudoku, Minesweeper, Tetris, \ldots
\end{itemize}
\bigskip\pause

\alert{\PSpace} ist eine typische Klasse für Spiele, bei denen zwei Spieler
abwechselnd eine polynomielle Zahl an Zügen durchführen:
\begin{itemize}
\item Geography, Reversi, Tic-Tac-Toe, aber auch: Sokoban, \ldots
\end{itemize}
\bigskip\pause

\alert{\ExpTime} findet sich bei Spielen, bei denen man Züge rückgängig machen kann (polynomielles Spielbrett -- exponentiell viele Züge):
\begin{itemize}
\item Schach, Dame, Go, Stern-Halma, \ldots
\end{itemize}
\smallskip

In jedem Fall muss man (nichtendliche) Verallgemeinerungen der Spiele betrachten,
um die "`wahre"' Komplexität zu sehen\\ {\tiny(Menschen spielen nicht, indem sie
innerlich eine endliche Datenbank aller möglichen Stellungen konsultieren)}
\smallskip\pause

\narrowcentering{\large\redalert{Spiele sollten komplex sein, um lange zu motivieren}}


\end{frame}

\begin{frame}\frametitle{Und sonst so?}

Es gibt viele weitere Themen in der Komplexitätstheorie
\bigskip

$\leadsto$ siehe Vorlesung "`Complexity Theory"' (Wintersemester)
\bigskip

\ldots und allgemein die Angebote im Track \alert{"`Logical Modeling"'} des
internationalen Masterstudiengangs \alert{"`Computational Modeling and Simultation"'} an der TU Dresden

\end{frame}


\begin{frame}\frametitle{Zusammenfassung und Ausblick}

\Slang{TrueQBF} und \Slang{Geography} sind \PSpace-vollständig
\bigskip

Wir kennen die praktisch wichtigsten Komplexitätsklassen und jeweils typische Probleme
\bigskip

Spiele liefern interessante Beispiele komplexer Probleme
\bigskip

\anybox{yellow}{
Was erwartet uns als Nächstes?
\begin{itemize}
\item Einführung in die Prädikatenlogik
\item Entscheidbare logische Probleme
\item Unentscheidbare logische Probleme
\end{itemize}
}

\end{frame}



% \begin{frame}[t]\frametitle{Literatur und Bildrechte}
% 
% \alert{Literatur}\bigskip
% 
% \begin{itemize}
% \item Richard J. Lorentz:
% \emph{Creating Difficult Instances of the Post Correspondence Problem.}
% Computers and Games 2000: 214--228
% \item John J. O'Connor, Edmund F. Robertson: \emph{Emil Leon Post.} MacTutor History of Mathematics archive, University of St Andrews. \url{http://www-history.mcs.st-andrews.ac.uk/Biographies/Post.html}
% \end{itemize}
% 
% \bigskip\bigskip
% 
% \alert{Bildrechte}\bigskip
% 
% Folie \ref{frame_post}: gemeinfrei
% 
% \end{frame}


\end{document}
