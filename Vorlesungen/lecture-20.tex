\documentclass[aspectratio=1610,onlymath]{beamer}
% \documentclass[aspectratio=1610,onlymath,handout]{beamer}

\input{macros-lecture}

\defineTitle{20}{Resolution (2)}{5. Juli 2021}

\begin{document}

\maketitle

% \bgroup
% \setbeamercolor{background canvas}{bg=black}
% \frame[plain]{\label{frame_herbrand}\begin{center}\color{white}
% \includegraphics[height=6cm]{images/Herbrand.jpg}

% \LARGE
% Jacques Herbrand\medskip
% 
% \normalsize
% \tt 12.2.1908 -- 27.7.1931
% \end{center}}
% \egroup

\begin{frame}\frametitle{Der Resolutionsalgorithmus}

Resolutionsregel:
\[  \frac{
\{A_1,\ldots,A_n,L_1,\ldots,L_k\}\quad
\{\neg A'_1,\ldots,\neg A'_m,L'_1,\ldots,L'_\ell\}
}{\{L_1\sigma,\ldots,L_k\sigma,L'_1\sigma,\ldots,L'_\ell\sigma\}
}\]
falls $\sigma$ allgemeinster Unifikator von $\{A_1,\ldots,A_n, A'_1,\ldots, A'_m\}$ ist\bigskip

\emph{Algorithmus (Skizze):}
\begin{enumerate}[(1)]
\item Bilde Klauselform
\item Bilde systematisch Resolventen durch Resolution von Varianten bereits abgeleiteter Klauseln
\item Wiederhole (2), bis entweder $\bot$ erzeugt wird ("`unerfüllbar"') oder keine neuen Klauseln mehr entstehen\footnote{Dieser Fall ist eher ungewöhnlich: Meist entstehen bei erfüllbaren Theorien immer mehr neue Klauseln, ohne dass das Verfahren terminiert.}
\end{enumerate}

\end{frame}

\begin{frame}\frametitle{Vollständigkeit und Korrektheit}

\theobox{\emph{Resolutionssatz:} Sei $F$ eine prädikatenlogische Formel und $\mathcal{K}_i$ ($i\geq 0$) die vom Resolutionsalgorithmus ermittelten Klauselmengen. Dann sind die folgenden Aussagen äquivalent:
\begin{itemize}
\item $F$ ist unerfüllbar
\item Es gibt ein $\ell\geq 0$ mit $\bot\in\mathcal{K}_\ell$
\end{itemize}}

\begin{itemize}
\item Korrektheit hatten wir bereits gezeigt
\item Vollständigkeit steht noch aus
\end{itemize}

\end{frame}

\bgroup
\setbeamercolor{background canvas}{bg=black}
\frame[plain]{\label{frame_herbrand}\begin{center}\color{white}
\includegraphics[height=6cm]{images/Herbrand.jpg}

\LARGE
Jacques Herbrand\medskip

\normalsize
\tt 12.2.1908 -- 27.7.1931
\end{center}}
\egroup

\bgroup
\setbeamercolor{background canvas}{bg=white}
\frame[plain]{\label{frame_jasny}\begin{center}\color{black}
\includegraphics[height=6cm]{images/Natascha_Artin_Brunswick.jpg}

\LARGE
Natasha Artin Brunswick\medskip

\normalsize
\tt 11.6.1909 -- 3.2.2003
\end{center}}
\egroup

\begin{frame}\frametitle{Syntax vs. Semantik}

Bei Herbrand-Interpretationen kann man semantische Elemente (wie sie in Zuweisungen vorkommen) durch syntaktische Elemente (wie sie in Substitutionen vorkommen) ausdrücken:\medskip

\theobox{\emph{Lemma:} Für jede Herbrand-Interpretation $\Inter$, jede Zuweisung $\Zuweisung$ für $\Inter$, jeden Term $t\in\Delta^\Inter$ und jede Formel $F$ gilt:
\[ \Inter,\Zuweisung\{x\mapsto t\} \models F \qquad \text{gdw.}\qquad \Inter,\Zuweisung \models F\{x\mapsto t\}\]}

(ohne Beweis; einfach)
\bigskip

{\color{devilscss}\footnotesize \emph{Anmerkung:} Man kann ein entsprechendes Resultat auch für Nicht-Herbrand-Interpretationen zeigen. Dann muss man einfach den Term auf der linken Seite durch $t^{\Inter,\Zuweisung}$ ersetzen.}\bigskip

\theobox{\emph{Satz:} Ein Satz $F$ in Skolemform ist genau dann erfüllbar, wenn $F$ ein
Herbrand-Modell hat.}

\end{frame}

% \begin{frame}\frametitle{Erfüllbar + Skolem = Erfüllbarkeit bei Herbrand}
% 
% \theobox{\emph{Satz:} Ein Satz $F$ in Skolemform ist genau dann erfüllbar, wenn $F$ ein
% Herbrandmodell hat.}\pause
% 
% \emph{Beweis:} $(\Leftarrow)$ ist klar, da Herbrandmodelle auch Modelle sind.\bigskip\pause
% 
% $(\Rightarrow)$ Sei $\Inter\models F$ ein Modell für $F$. Wir definieren eine Herbrandinterpretation $\Jnter$ indem wir festlegen:
% \begin{itemize}
% \item $p^\Jnter=\{\tuple{t_1,\ldots,t_n}\mid\tuple{t_1^\Inter,\ldots,t_n^\Inter}\in p^\Inter\}$\\
% {\footnotesize Anm.: $t_i$ sind variablenfrei, daher ist $t_i^\Inter$ wohldefiniert}
% \end{itemize}
% Behauptung: \alert{$\Jnter$ ist ein Herbrandmodell von $F$}
% 
% \end{frame}
% 
% \begin{frame}\frametitle{Beweis (Fortsetzung)}
% 
% \emph{Behauptung:} \alert{$\Jnter$ ist ein Herbrandmodell von $F$}\bigskip
% 
% $F$ hat die Form $\forall x_1,\ldots, x_n.G$, wobei $G$ quantorenfrei ist.\pause
% \begin{itemize}
% \item Aus $\Inter\models F$ folgt also $\Inter,\Zuweisung\models G$ für jede Zuweisung $\Zuweisung$ für $\Inter$\pause
% \item Speziell gilt also für alle $t_1,\ldots,t_n\in\Delta_F$: $\Inter,\{x_1\mapsto t_1^{\Inter},\ldots,x_n\mapsto t_n^{\Inter}\}\models G$\pause
% \item Daraus folgt: $\Inter\models G\{x_1\mapsto t_1,\ldots,x_n\mapsto t_n\}$  (analog zu Lemma)\pause
% \item Daraus folgt: $\Jnter\models G\{x_1\mapsto t_1,\ldots,x_n\mapsto t_n\}$\\
% {\footnotesize(Für Atome $G$ direkt aus Definition; die Aussage kann leicht auf größere Boolsche Verknüpfungen von Atomen verallgemeinert werden -- formal durch strukturelle Induktion)}\pause
% \item Es folgt: $\Jnter,\{x_1\mapsto t_1,\ldots,x_n\mapsto t_n\}\models G$ (Lemma)
% \end{itemize}
% Der Schluss gilt für alle $t_1,\ldots,t_n\in\Delta_F$, d.h. $\Jnter\models F$.\qed
% 
% \end{frame}
% 

\sectionSlide{Prädikatenlogisches Schließen mit Aussagenlogik}

\begin{frame}\frametitle{Herbrand-Expansionen}

\defbox{Die \redalert{Herbrand-Expansion} $HE(F)$ einer Formel $F=\forall x_1,\ldots, x_n.G$ in Skolemform ist die Menge:
\[ HE(F)\defeq\big\{ G\{x_1\mapsto t_1,\ldots,x_n\mapsto t_n\}\mid t_1,\ldots,t_n\in\Delta_F\big\}\]
}

$HE(F)$ ist also die (möglicherweise unendliche) Menge von \ghost{variablen-}\\ freien Sätzen, die in Herbrand-Modellen von $F$ gelten müssten.
\pause\bigskip

\redalert{Quantorenfreie Sätze = aussagenlogische Formeln:}
\begin{itemize}
\item $HE(F)$ enthält Formeln ohne Variablen, d.h. Boolsche Kombinationen geschlossener Atome
\item Geschlossene Atome können unabhängig voneinander wahr oder falsch sein, egal wie ihre genaue Struktur aussieht
\item Wir können sie also als "`ungewöhnlich benannte"' aussagenlogische Atome auffassen und die gesamte Formel aussagenlogisch interpretieren
\end{itemize}
\alert{$\leadsto$ $HE(F)$ als aussagenlogische Theorie}


\end{frame}

\begin{frame}\frametitle{Gödel, Herbrand, Skolem}

\theobox{\emph{Satz von Gödel, Herbrand \& Skolem:} Eine Formel $F$ in Skolemform ist genau dann erfüllbar, wenn $HE(F)$ aussagenlogisch erfüllbar ist.}\pause

\emph{Beweis:} Wir zeigen, dass $F=\forall x_1,\ldots,x_n.G$ genau dann ein Herbrand-Modell hat, wenn $HE(F)$ aussagenlogisch erfüllbar ist:\pause

\begin{itemize}
\item $\Inter$ ist ein Herbrand-Modell von $F$\pause
\item gdw. $\Inter,\{x_1\mapsto t_1,\ldots,x_n\mapsto t_n\}\models G$ für alle $t_1,\ldots,t_n\in\Delta_F$\pause
\item gdw. $\Inter\models G\{x_1\mapsto t_1,\ldots,x_n\mapsto t_n\}$ für alle $t_1,\ldots,t_n\in\Delta_F$ (Lemma)\pause
\item gdw. für alle $H\in HE(F)$ gilt $\Inter\models H$\pause
\item gdw. $\Inter$ als aussagenlogisches Modell für $HE(F)$ angesehen werden kann.\qed
\end{itemize}

\end{frame}

\begin{frame}\frametitle{Satz von Herbrand}

Als Korollar der gezeigten Ergebnisse erhalten wir ein wichtiges Resultat:

\theobox{\emph{Satz:} Eine Formel $F$ in Skolemform ist genau dann unerfüllbar, wenn
eine endliche Teilmenge von $HE(F)$ aussagenlogisch unerfüllbar ist.}

\pause\emph{Beweis:} Die Kontraposition des Satzes von Gödel, Herbrand \& Skolem besagt:\medskip

Eine Formel $F$ in Skolemform ist genau dann unerfüllbar, wenn $HE(F)$ aussagenlogisch unerfüllbar ist.\bigskip

Der Satz folgt nun, weil jede unerfüllbare aussagenlogische Formelmenge eine endliche
Teilmenge hat, die unerfüllbar ist: \alert{Kompaktheit der Aussagenlogik}, ohne Beweis\\[1ex]
{\tiny
\textcolor{devilscss}{Das Ergebnis kann aus der Vollständigkeit der Verallgemeinerung aussagenlogischer Resolution auf unendliche Modelle gefolgert werden (siehe Formale Systeme, WS 2020/2021, Vorlesung 23): wenn die leere Klausel endlich abgeleitet werden kann, dann nutzt man dazu nur endlich viele Klauseln der Eingabe; wenn die leere Klausel nicht endlich abgeleitet werden kann, dann erhält man aus der unendlichen Menge aller möglichen Ableitungen ein Modell, analog zum endlichen Fall.}

}\qed

\end{frame}

\begin{frame}\frametitle{Prädikatenlogik semi-entscheiden}

Das Ergebnis Herbrands ermöglicht bereits einen naiven Algorithmus zur Semi-Entscheidung von Unerfüllbarkeit in der Prädikatenlogik:\bigskip

\codebox{%
\emph{Gegeben:} Eine Formel $F$
\begin{itemize}
\item Wandle $F$ in Skolemform $F'$ um
\item Definiere eine Reihenfolge der Formeln in $HE(F')$: $F_1,F_2,F_3,\ldots$
\item Für alle $i\geq 1$:
	\begin{itemize}
	\item Prüfe ob die endliche Menge $\{F_1,\ldots, F_i\}$ aussagenlogisch unerfüllbar ist
	\item Falls ja, dann gib "`unerfüllbar"' aus; andernfalls fahre fort
	\end{itemize}
\end{itemize}}\medskip

Offenbar ist das \redalert{kein praktischer Algorithmus}, aber er zeigt Semi-Entscheidbarkeit

\end{frame}

\sectionSlide{Vollständigkeit der Resolution}

\begin{frame}\frametitle{Ansatz}

\alert{Herbrands Satz liefert uns auch eine Strategie zum Beweis der Vollständigkeit des Resolutionsalgorithmus}\bigskip

\emph{Wir wissen bereits:}
\begin{itemize}
\item Unerfüllbarkeit einer Klauselmenge zeigt sich in der Unerfüllbarkeit ihrer Herbrand-Expansion
\item Die Unerfüllbarkeit der Herbrand-Expansion kann man mit aussagenlogischer Resolution beweisen
\item Prädikatenlogische Resolution verallgemeinert aussagenlogische Resolution, indem wir direkt mit Klauseln arbeiten, die noch Variablen enthalten
\end{itemize}\pause

\emph{Frage:} Kann man alle Schlüsse, die man auf expandierten Formeln aussagenlogisch erzeugen kann, auch direkt prädikatenlogisch (mit Variablen) erhalten?

\end{frame}

\begin{frame}\frametitle{Lifting-Lemma}

Wir zeigen: Ja, jeder aussagenlogische Schluss (auf der Expansion) kann auf einen prädikatenlogischen Schluss (auf den Klauseln mit Variablen) "`angehoben"' werden.\medskip

\theobox{\emph{Satz (Lifting-Lemma):} Seien $K_1$ und $K_2$ prädikatenlogische Klauseln mit
Grundinstanzen $K'_1=K_1\sigma$ und $K'_2=K_2\sigma$.${^1}$\bigskip

Wenn $R'$ eine (aussagenlogische) Resolvente von $K'_1$ und $K'_2$ ist, dann gibt es
eine prädikatenlogische Resolvente $R$ von $K_1$ und $K_2$, welche $R'$ als Grundinstanz hat.
}

\color{devilscss}
{\footnotesize ${^1}$ Die Verwendung derselben Substitution für $K'_1$ und $K'_2$ ist keine Einschränkung, da wir durch Variantenbildung sicherstellen können, dass $K_1$ und $K_2$ keine Variablen gemein haben.

}

\end{frame}

\begin{frame}\frametitle{Lifting-Lemma: Beweis}

\theobox{\emph{Satz (Lifting-Lemma):} Seien $K_1$ und $K_2$ prädikatenlogische Klauseln mit
Grundinstanzen $K'_1=K_1\sigma$ und $K'_2=K_2\sigma$.\smallskip

Wenn $R'$ eine (aussagenlogische) Resolvente von $K'_1$ und $K'_2$ ist, dann gibt es
eine prädikatenlogische Resolvente $R$ von $K_1$ und $K_2$, welche $R'$ als Grundinstanz hat.
}

\emph{Beweis:} Sei $A'\in K'_1$ das (geschlossene) Atom, über das resolviert wurde, d.h. $\neg A'\in K'_2$.\smallskip\pause

Sei $\mathcal{A}_1\defeq\{A\mid A\in K_1,\;A\sigma=A'\}$ und $\mathcal{A}_2\defeq\{A\mid \neg A\in K_2, A\sigma=A'\}$.\smallskip\pause

Dann ist $\sigma$ ein Unifikator für $\mathcal{A}_1\cup\mathcal{A}_2$.\\ Also hat $\mathcal{A}_1\cup\mathcal{A}_2$ einen allgemeinsten Unifikator $\theta$.\smallskip\pause

Sei $R$ die Resolvente von $K_1$ und $K_2$ bzgl. $\theta$.\smallskip\pause

Dann enthalten $R'$ und $R$ Instanzen der gleichen Literale, d.h. sie sind von der Form
$R'=\{L_1\sigma,\ldots,L_n\sigma\}$ und $R=\{L_1\theta,\ldots,L_n\theta\}$\smallskip\pause

Da $\theta$ allgemeinster Unifikator ist, gibt es $\lambda$ mit $\sigma=\theta\circ\lambda$
und es gilt: $R\lambda=\{L_1\theta\lambda,\ldots,L_n\theta\lambda\}=\{L_1\sigma,\ldots,L_n\sigma\}=R'$\qed


\end{frame}


\begin{frame}[t]\frametitle{Vollständigkeit der Resolution (1)}

\theobox{\emph{Resolutionssatz:} Sei $F$ eine prädikatenlogische Formel und $\mathcal{K}_i$ ($i\geq 0$) die vom Resolutionsalgorithmus ermittelten Klauselmengen. Dann sind die folgenden Aussagen äquivalent:
\begin{itemize}
\item $F$ ist unerfüllbar
\item Es gibt ein $\ell\geq 0$ mit $\bot\in\mathcal{K}_\ell$
\end{itemize}}\pause

\emph{Beweis (Vollständigkeit):} Sei $F$ unerfüllbar
\begin{itemize}
\item Dann ist $HE(F)$ unerfüllbar
\item Dann gibt es eine (endliche) aussagenlogische Resolutionsableitung von $\bot$ aus $HE(F)$
\item Die Ableitung erzeugt eine endliche Folge von Klauseln: $K_1',K_2',\ldots,K_{m-1}',K_m'=\bot$
\item \alert{Behauptung:} Jede Klausel $K_i'$ ist Grundinstanz einer Klausel $K_i$ die in $\mathcal{K}_\ell$ vorkommt für ein $\ell\geq 0$.
\item Für $i=m$ folgt daraus der Satz, denn $K_m'=\bot$ kann nur Grundinstanz von $\bot$ sein, d.h. $\bot\in\mathcal{K}_\ell$ für ein $\ell\geq 0$.
\end{itemize}

\end{frame}

\begin{frame}[t]\frametitle{Vollständigkeit der Resolution (2)}

% \theobox{Resolutionssatz: Sei $F$ eine prädikatenlogische Formel und $\mathcal{K}_i$ ($i\geq 0$) die vom Resolutionsalgorithmus ermittelten Klauselmengen. Dann sind die folgenden Aussagen äquivalent:
% \begin{itemize}
% \item $F$ ist unerfüllbar
% \item Es gibt ein $\ell\geq 0$ mit $\bot\in\mathcal{K}_\ell$
% \end{itemize}}

\emph{Beweis (Vollständigkeit):} \alert{Behauptung:} Jede Klausel $K_i'$ ist Grundinstanz einer Klausel $K_i$ die in $\mathcal{K}_\ell$ vorkommt für ein $\ell\geq 0$.
\medskip\pause

Aussage klar für $K'_i\in HE(F)$: in diesem Fall ist $K'_i$ Grundinstanz einer Klausel $K_i$ in der Klauselform von $F$ und in $\mathcal{K}_0$\medskip\pause

Restlicher Beweis durch Induktion über $i$:
\begin{itemize}
\item Induktionsannahme: Die Aussage gilt für alle $j<i$\pause
\item $K'_i$ ist Resolvente zweier Klauseln $K'_a$ und $K'_b$ mit $a,b<i$\pause
\item Laut Hypothese sind $K'_a$ und $K'_b$ also Instanzen von Klauseln $K_a$ und $K_b$ in einer Menge $\mathcal{K}_\ell$\pause
\item Laut Lifting-Lemma ist demnach $K'_i$ ebenfalls die Instanz einer Klausel $K_i$, die durch Resolution aus $K_a$ und $K_b$ entsteht\pause
\item Diese Resolvente $K_i$ ist also ebenfalls in einer Menge der Form $\mathcal{K}_{\ell'}$\qed
\end{itemize}

\end{frame}

\begin{frame}\frametitle{Kompaktheit}

Die Existenz von vollständigen und korrekten logischen Schließverfahren wie Resolution
ist eng verwandt mit einer grundsätzlichen Eigenschaft der Prädikatenlogik:\bigskip

\theobox{\emph{Satz (Endlichkeitssatz, Kompaktheitssatz):} Falls eine unendliche Menge prädikatenlogischer Sätze $\mathcal{T}$ eine logische Konsequenz $F$ hat, so ist $F$ auch Konsequenz einer endlichen Teilmenge von $\mathcal{T}$.}\pause

\small
\emph{Beweis:} Die gegebene logische Konsequenz ist gleichbedeutend damit, dass
$\mathcal{T}\cup\{\neg F\}$ unerfüllbar ist.\medskip

Laut Resolutionssatz (Vollständigkeit) kann die Unerfüllbarkeit von $\mathcal{T}\cup\{\neg F\}$ nach endlich vielen Schritten durch Ableitung der leeren Klausel nachgewiesen werden.\medskip

Dabei können nur endlich viele Klauseln aus der Klauselform von $\mathcal{T}\cup\{\neg F\}$ verwendet worden sein. Laut Resolutionssatz (Korrektheit) folgt die Konsequenz also bereits aus einer endlichen Teilmenge von $\mathcal{T}$.\qed

\end{frame}

\begin{frame}\frametitle{Die Grenzen der Prädikatenlogik}

Kompaktheit zeigt uns auch erste Grenzen der Prädikatenlogik auf.
\smallskip

\defbox{Eine logische Formel $F$ mit zwei freien Variablen $x$ und $y$ drückt den \redalert{transitiven Abschluss} einer binären Relation $r$ aus, wenn in jeder Interpretation $\Inter$ und für alle $\delta_1,\delta_2\in\Delta^\Inter$ gilt:\\[1ex]
\narrowcentering{$ \Inter,\{x\mapsto \delta_1,y\mapsto\delta_2\}\models F \quad\text{gdw.}\quad \tuple{\delta_1,\delta_2}\in (r^\Inter)^* $}
}\pause

\theobox{Satz: Es gibt keine prädikatenlogische Formel, die den transitiven Abschluss einer binären Relation ausdrückt.}\pause

\emph{Beweis:} Angenommen es gäbe so eine Formel $F$.\medskip

Dann ist die folgende unendliche Theorie unerfüllbar:
\[\begin{array}{rll}
\big\{ & F\{x\mapsto a,y\mapsto b\},\neg r(a,b), \neg \exists x_1.(r(a,x_1)\wedge r(x_1,b)),\\
& \neg\exists x_1,x_2.(r(a,x_1)\wedge
r(x_1,x_2)\wedge r(x_2,b)),\ldots & \big\}
\end{array}\]
Aber jede endliche Teilmenge der Theorie ist erfüllbar.
Die Existenz der Theorie würde also dem Endlichkeitssatz widersprechen.\qed

\end{frame}

\begin{frame}\frametitle{Zusammenfassung und Ausblick}

Die prädikatenlogische Resolution ist ein vollständiges und korrektes Verfahren für die Unerfüllbarkeit logischer Formeln\bigskip

In gewissem Sinne ist Prädikatenlogik eine Kurzschreibweise für möglicherweise unendliche aussagenlogische Theorien
\bigskip

% Bei Beschränkung auf endliche Modelle gibt es kein vollständiges und korrektes Verfahren zum logischen Schließen -- dafür wird Erfüllbarkeit semi-entscheidbar\bigskip

% Auswertungsproblem auf endlichen Modellen = Anfragebeantwortung in Datenbanken

\anybox{yellow}{
Was erwartet uns als Nächstes?
\begin{itemize}
\item Endliche Modelle und Datenbanken
\item Datalog
\item Gödel
\end{itemize}
}

\end{frame}


\begin{frame}[t]\frametitle{Bildrechte}

Folie \ref{frame_herbrand}: Fotografie von Natasha Artin Brunswick, 1931, CC-By 3.0

Folie \ref{frame_jasny}: Fotografie von Thomas Artin, 2000, CC-By 3.0

\end{frame}


\end{document}
