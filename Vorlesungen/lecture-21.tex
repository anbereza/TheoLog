\documentclass[aspectratio=1610,onlymath]{beamer}
% \documentclass[aspectratio=1610,onlymath,handout]{beamer}

\input{macros-lecture}

\defineTitle{21}{Endliche Modelle und Datenbanken}{8. Juli 2021}

\begin{document}

\maketitle

% \begin{frame}\frametitle{Rückblick: Kompaktheit}
% 
% Die Existenz von vollständigen und korrekten logischen Schließverfahren wie Resolution
% ist eng verwandt mit einer grundsätzlichen Eigenschaft der Prädikatenlogik:\bigskip
% 
% \theobox{\emph{Satz (Endlichkeitssatz, Kompaktheitssatz):} Falls eine unendliche Menge prädikatenlogischer Sätze $\mathcal{T}$ eine logische Konsequenz $F$ hat, so ist $F$ auch Konsequenz einer endlichen Teilmenge von $\mathcal{T}$.}\pause
% 
% \small
% \emph{Beweis:} Die gegebene logische Konsequenz ist gleichbedeutend damit, dass
% $\mathcal{T}\cup\{\neg F\}$ unerfüllbar ist.\medskip
% 
% Laut Resolutionssatz (Vollständigkeit) kann die Unerfüllbarkeit von $\mathcal{T}\cup\{\neg F\}$ nach endlich vielen Schritten durch Ableitung der leeren Klausel nachgewiesen werden.\medskip
% 
% Dabei können nur endlich viele Klauseln aus der Klauselform von $\mathcal{T}\cup\{\neg F\}$ verwendet worden sein. Laut Resolutionssatz (Korrektheit) folgt die Konsequenz also bereits aus einer endlichen Teilmenge von $\mathcal{T}$.\qed
% 
% \end{frame}
% 
% 
% \begin{frame}\frametitle{Die Grenzen der Prädikatenlogik}
% 
% Kompaktheit zeigt uns auch erste Grenzen der Prädikatenlogik auf.
% \smallskip
% 
% \defbox{Eine logische Formel $F$ mit zwei freien Variablen $x$ und $y$ drückt den \redalert{transitiven Abschluss} einer binären Relation $r$ aus, wenn in jeder Interpretation $\Inter$ und für alle $\delta_1,\delta_2\in\Delta^\Inter$ gilt:\\[1ex]
% \narrowcentering{$ \Inter,\{x\mapsto \delta_1,y\mapsto\delta_2\}\models F \quad\text{gdw.}\quad \tuple{\delta_1,\delta_2}\in (r^\Inter)^* $}
% }\pause
% 
% \theobox{\emph{Satz:} Es gibt keine prädikatenlogische Formel, die den transitiven Abschluss einer binären Relation ausdrückt.}\pause
% 
% \emph{Beweis:} Angenommen es gäbe so eine Formel $F$.\medskip
% 
% Dann ist die folgende unendliche Theorie unerfüllbar:
% \[\begin{array}{rll}
% \big\{ & F\{x\mapsto a,y\mapsto b\},\neg r(a,b), \neg \exists x_1.(r(a,x_1)\wedge r(x_1,b)),\\
% & \neg\exists x_1,x_2.(r(a,x_1)\wedge
% r(x_1,x_2)\wedge r(x_2,b)),\ldots & \big\}
% \end{array}\]
% Aber jede endliche Teilmenge der Theorie ist erfüllbar.
% Die Existenz der Theorie würde also dem Endlichkeitssatz widersprechen.\qed
% 
% \end{frame}


\sectionSlide{Endliche Modelle}

\begin{frame}\frametitle{Endlichkeit von Modellen}

\alert{Löwenheim-Skolem}: Jede erfüllbare Formel hat ein abzählbar großes Modell
\bigskip

\redalert{Kann man dies noch verstärken? Hat jede erfüllbare Formel vielleicht sogar ein endliches Modell?}\pause\bigskip

\alert{Nein! Prädikatenlogik kann unendliche Modelle erzwingen:}\medskip

\examplebox{\emph{Beispiel:}\vspace{-2ex}
\begin{align*}
\forall x.&(\textsf{Mensch}(x) \to\exists y.(\textsf{hatMutter}(x,y)\wedge\textsf{Mensch}(y))) \\
\forall x,y.&(\textsf{hatMutter}(x,y) \to\textsf{hatVorfahre}(x,y))\\
\forall x,y,z.&((\textsf{hatVorfahre}(x,y)\wedge\textsf{hatVorfahre}(y,z))\to\textsf{hatVorfahre}(x,z))\\
\forall x.&\neg \textsf{hatVorfahre}(x,x)
\end{align*}
Diese Theorie ist erfüllbar, aber hat nur unendliche Modelle.\\ (Kontrollfrage: Warum?)
}

\end{frame}

\begin{frame}\frametitle{Logik über endlichen Modellen}

Sind unendliche Modelle in der Praxis überhaupt wünschenswert?\\
Geht es auch endlich?
\bigskip\pause

\defbox{\redalert{Prädikatenlogik mit endlichen Modellen} verwendet die gleiche Syntax und Semantik wie Prädikatenlogik allgemein, aber mit der zusätzlichen Bedingung, dass die Domäne von Interpretationen endlich sein muss.}\pause

\emph{Monotonie (Rückblick):} weniger Modelle $=$ mehr Konsequenzen

\examplebox{\emph{Beispiel:}\vspace{-2ex}
\begin{align*}
\forall x.&(\textsf{Mensch}(x) \to\exists y.(\textsf{hatVorfahre}(x,y)\wedge\textsf{Mensch}(y))) \\
% \forall x,y.&(\textsf{hatMutter}(x,y) \to\textsf{hatVorfahre}(x,y))\\
\forall x,y,z.&((\textsf{hatVorfahre}(x,y)\wedge\textsf{hatVorfahre}(y,z))\to\textsf{hatVorfahre}(x,z))
\end{align*}\vspace{-4ex}

Diese Theorie ist in der Prädikatenlogik mit endlichen Modellen erfüllbar, aber jedes endliche Modell muss einen $\textsf{hatVorfahre}$-Zyklus enthalten. Daher folgt
$\exists x.\textsf{hatVorfahre}(x,x)$, obwohl dies in der allgemeinen Prädikatenlogik keine Konsequenz wäre.
}

\end{frame}

\begin{frame}\frametitle{Erfüllbarkeit wird semi-entscheidbar}

Die Bezeichnung der Elemente einer Interpretationsdomäne ist irrelevant -- für die Wahrheit von Sätzen kommt es nur darauf an, wie Konstanten und Prädikatsymbole interpretiert werden.\bigskip\pause

\anybox{strongyellow}{\redalert{Erfüllbarkeitstest}\\[1ex]
\emph{Gegeben:} Ein Satz $F$
\begin{itemize}
\item Betrachte systematisch alle endlichen Interpretationen der Symbole in $F$ (z.B. geordnet nach aufsteigender Größe der Domäne)
\item Prüfe für jedes Modell $\Inter$, ob $\Inter\models F$ gilt:
\begin{itemize}
\item Falls ja, dann gib aus "`erfüllbar"'
\item Falls nein, dann fahre mit nächster Interpretation fort
\end{itemize}
\end{itemize}
}

Es ist leicht zu sehen, dass dieser Algorithmus die Erfüllbarkeit in endlichen Modellen semi-entscheidet.

\end{frame}

\begin{frame}\frametitle{Endlich = einfach?}\pause

Trotzdem bleibt logisches Schließen schwer:

\theobox{\emph{Satz von Trakhtenbrot:} Logisches Schließen (Erfüllbarkeit, Allgemeingültigkeit, logische Konsequenz) in der Prädikatenlogik mit endlichen Modellen ist unentscheidbar.}

(ohne Beweis; mehr dazu in der Vorlesung \alert{Database Theory})\pause\bigskip

\theobox{\emph{Korollar:} Es gibt kein vollständiges und korrektes Beweissystem für Prädikatenlogik mit endlichen Modellen.}

\emph{Beweis:} Angenommen es gäbe ein solches System.\pause{} Dann wäre logische Konsequenz und speziell auch Unerfüllbarkeit semi-entscheidbar.\bigskip\pause

Wir wissen, dass Erfüllbarkeit ebenfalls semi-entscheidbar ist.\bigskip\pause

Zusammen ergäbe sich also ein Entscheidungsverfahren für logisches Schließen -- Widerspruch zu Trakhtenbrot.
\qed

\end{frame}

\sectionSlide{Endliche Modelle in der Praxis}

\begin{frame}\frametitle{Wozu endliche Modelle}

In gewisser Weise ist Schließen mit endlichen Modellen also schwerer als mit unendlichen, weil man statt logischer Konsequenz nunmehr nur Nicht-Konsequenz semi-entscheiden kann
\bigskip

Trotzdem sind endliche Interpretationen in der Informatik praktisch relevant:\medskip

\anybox{purple}{Eine endliche Interpretation $\Inter$ ist (im Wesentlichen) das gleiche wie eine \redalert{relationale Datenbankinstanz}.
}
\emph{Intuition:}
\begin{itemize}
\item Prädikatsymbole $p$ bezeichnen Tabellen
\item Relationen $p^\Inter$ entsprechen den in der Datenbank gespeicherten Tabelleninhalten
\end{itemize}

\end{frame}

\begin{frame}\frametitle{Benannte Parameter}

Relationale Datenbanken verwenden \redalert{Namen für die Parameter} (Spalten) in Relationen,
anstatt sie mittels Reihenfolge zu adressieren:\bigskip

\scriptsize

\begin{tabular}[t]{|l|l|}
\multicolumn{2}{@{}l}{linien:}\\
\hline
\textbf{Linie} & \textbf{Typ} \\
\hline
85 & Bus \\\hline
3 & Tram \\\hline
F1 & Fähre \\\hline
\ldots & \ldots\\\hline
\end{tabular}
\hspace{15mm}
\begin{tabular}[t]{|l|l|l|}
\multicolumn{2}{@{}l}{haltestellen:}\\
\hline
\textbf{SID} & \textbf{Name} & \textbf{Rollstuhl}\\
\hline
17 & Hauptbahnhof & true\\\hline
42 & Helmholtzstr. & true\\\hline
57 & Stadtgutstr. & true\\\hline
123 & Gustav-Freytag-Str. & false\\\hline
\ldots & \ldots & \ldots\\\hline
\end{tabular}

\begin{tabular}[t]{|l|l|l|}
\multicolumn{2}{@{}l}{verbindung:}\\
\hline
\textbf{Von} & \textbf{Zu} & \textbf{Linie}\\
\hline
57  & 42 & 85 \\\hline
17  & 789 & 3 \\\hline
\ldots & \ldots & \ldots\\\hline
\end{tabular}
\hspace{0.5cm}
% 
% \medskip
\ghost{
\begin{minipage}[t]{7cm}
\vspace{0.0cm}
Die einfache Arität der Prädikatenlogik wird durch ein\\ \redalert{Schema} mit Namen (und oft auch Datentypen) ersetzt:
\begin{itemize}
\item linien[Linie:string, Typ:string]
\item haltestellen[SID:int, Halt:string, Rollstuhl:bool]
\item verbindung[Von:int, Zu:int, Linie:string]
\end{itemize}
\end{minipage}}

\end{frame}

\begin{frame}\frametitle{Formeln = Anfragen}

Benannt oder nicht -- sofern die Parameter eine definierte Reihenfolge haben,
kann man sie mit normalen prädikatenlogischen Atomen adressieren.\medskip

\examplebox{\emph{Beispiel:} Die Formel \[Q=\exists z_{\text{Linie}}.(\textsf{verbindung}(x_\text{Von},x_\text{Zu}, z_{\text{Linie}})\wedge \textsf{linien}(z_{\text{Linie}},x_{\text{Typ}}))\]
hat drei freie Variablen. Für eine gegebene Datenbankinstanz (endliche Interpretation) $\Inter$ bedeutet $\Inter,\{x_\text{Von}\mapsto\delta_1,x_\text{Zu}\mapsto\delta_2,x_\text{Typ}\mapsto\delta_3\}\models Q$, dass es in der Datenbank eine Verbindung von $\delta_1$ nach $\delta_2$ vom Typ $\delta_3$ gibt.
}\pause

Das Beispiel illustriert:
\begin{center}
\large
\redalert{Formeln (ev. mit freien Variablen) = Datenbankanfragen}\\[1ex]
\redalert{Erfüllende Zuweisungen = Anfrage-Ergebnisse}\\
\end{center}
\end{frame}

\begin{frame}\frametitle{Logik und Datenbanken}

\alert{Relationale Datenbankinstanzen = Endliche Interpretationen}
\begin{itemize}
\item Tabellen(namen) entsprechen Prädikatssymbolen
\item Kleinere syntaktische Unterschiede (benannte vs. geordnete Parameter)
\end{itemize}

\alert{Relationale Datenbankanfragen = Prädikatenlogische Formeln}
\begin{itemize}
\item Zuweisungen $\Zuweisung$ zu freien Variablen als mögliche Anfrageergebnisse
\item $\Inter,\Zuweisung\models Q$ bedeutet: $\Zuweisung$ ist Ergebnis der Anfrage $Q$ auf Datenbankinstanz $\Inter$
% \item Prädikatenlogik kann die selben Anfragefunktionen ausdrücken wie normales SQL
\end{itemize}


\end{frame}


\begin{frame}\frametitle{Prädikatenlogik $\approx$ SQL}

\redalert{Was ist eine Datenbank-Anfrage?}
\begin{itemize}
\item Syntax: Eine Anfrage $Q$ ist ein Wort aus einer Anfragesprache
\item Semantik: Jede Anfrage $Q$ definiert eine Anfragefunktion $f_Q$, \ghost{die}\\ für jede Datenbankinstanz $\Inter$ eine Ergebnisrelation $f_Q(\Inter)$ liefert
\end{itemize}\pause

\examplebox{\emph{Beispiel:} für eine prädikatenlogische Formel $Q$ mit freien Variablen $x_1,\ldots,x_n$ ist $f_Q$ die Funktion, die $\Inter$ auf die Relation $f_Q(\Inter)=\{\tuple{\delta_1,\ldots,\delta_n}\mid \Inter,\{x_1\mapsto\delta_1,\ldots,x_n\mapsto\delta_n\}\models Q\}$ abbildet.
}\pause

Mit so einer allgemeinen Definition kann man sehr unterschiedliche Anfragesprachen über ihre Anfragefunktion vergleichen
\smallskip

\theobox{\emph{Satz:} Die Menge der durch prädikatenlogische Formeln $Q$ darstellbaren Anfragefunktionen $f_Q$ ist genau die Menge der Anfragefunktionen, die durch einfache SQL-Anfragen darstellbar sind.}

\tiny "`einfaches SQL"': Relationale Algebra, der Kern von SQL; SELECT, JOIN, UNION, MINUS, aber keine komplexeren Features wie WITH RECURSIVE etc. Außerdem keine Datentypen, da wir diese in Logik nicht eingeführt haben.

\end{frame}

\begin{frame}\frametitle{Relationale Algebren}

Datenbankanfragen werden oft in \redalert{relationaler Algebra} dargestellt,
bei der man Relationen mit Operationen zu einem Anfrageergebnis kombiniert
\bigskip

\examplebox{\emph{Beispiel:} Die Anfrage \[Q=\exists z_{\text{Linie}}.(\textsf{verbindung}(x_\text{Von},x_\text{Zu}, z_{\text{Linie}})\wedge \textsf{linien}(z_{\text{Linie}},x_{\text{Typ}}))\]
entspricht einer (natürlichen) Join-Operation ($\wedge$) mit anschließender Projektion ($\exists$):
\[ \pi_{\text{Von},\text{Zu},\text{Typ}}(\textsf{verbindung}\bowtie\textsf{linien})  \]
}\medskip

% \footnotesize
\emph{Anmerkung:} SQL hat noch einen leicht anderen Stil. Variablen stehen dort für ganze Tabellenzeilen und man verwendet Notation der Form "`$\textsf{linien}.\text{Typ}$"', um auf deren Einträge zuzugreifen ("`Tuple-Relational Calculus"'). Das ändert an der Ausdrucksstärke nichts.

\end{frame}

\begin{frame}\frametitle{Anfragebeantwortung als Model Checking}

\emph{Erkenntnis:} Die wesentliche Berechnungsaufgabe bei der Beantwortung von Datenbankabfragen ist das folgende Entscheidungsproblem:\bigskip

\defbox{Das \redalert{Auswertungsproblem} (\alert{Model Checking}) der Prädikatenlogik lautet wie folgt:\\[1ex]
\emph{Gegeben:} Eine Formel $Q$ mit freien Variablen $x_1,\ldots,x_n$; eine endliche Interpretation $\Inter$; Elemente $\delta_1,\ldots,\delta_n\in\Delta^\Inter$\\
\emph{Frage:} Gilt $\Inter,\{x_1\mapsto\delta_1,\ldots,x_n\mapsto\delta_n\}\models Q$?
}\bigskip\pause

Naive Methode der Anfragebeantwortung:
\begin{itemize}
\item Betrachte alle $(\Delta^\Inter)^n$ möglichen Ergebnisse
\item Entscheide jeweils das Auswertungsproblem
\end{itemize}

Praktisch relevante Frage:\medskip

\redalert{Wie schwer ist das Auswertungsproblem?}

\end{frame}


\newcommand{\mytab}{\texttt{~~}}
\begin{frame}\frametitle{Ein Algorithmus für das Auswertungsproblem}

Wir nehmen an, dass die Formel $F$ nur $\neg$, $\wedge$ und $\exists$ enthält
(durch Umformung möglich)\bigskip

\codebox{
$\texttt{function}~\text{Eval}(F,\Inter,\Zuweisung) :$
\smallskip

\begin{tabular}{rl}
01 & $\Scode{switch}~(F):$ \\[-1ex]
02 & $\mytab\Scode{case}~p(c_1,\ldots,c_n):~\texttt{return}~\tuple{c_1^{\Inter,\Zuweisung},\ldots,c_n^{\Inter,\Zuweisung}}\in p^\Inter$\\[-1ex]
03 & $\mytab\Scode{case}~\neg G:~\texttt{return}~\Scode{not~}\text{Eval}(G,\Inter,\Zuweisung)$\\[-1ex]
04 & $\mytab\Scode{case}~G_1\wedge G_2:~\texttt{return}~\text{Eval}(G_1,\Inter,\Zuweisung)\Scode{~and~}\text{Eval}(G_2,\Inter,\Zuweisung)$\\[-1ex]
05 & $\mytab\Scode{case}~\exists x.G:$\\[-1ex]
06 & $\mytab\mytab\Scode{for}~c\in\Delta^\Inter~:$\\[-1ex]
07 & $\mytab\mytab\mytab\Scode{if}~\text{Eval}(G\{x\mapsto c\},\Inter,\Zuweisung)~\Scode{then}~\texttt{return}~\Scode{true}$\\[-1ex]
08 & $\mytab\mytab\texttt{return}~\Scode{false}$\\[-1ex]
\end{tabular}
}

\emph{Anmerkung:} Wenn Konstanten $c$ in der Anfrage vorkommen, dann nimmt man
in der Regel an, dass $c^\Inter=c$ ist.\medskip

\emph{Anmerkung 2:} In der Praxis stimmt das nicht ganz. Insbesondere bei Verwendung von Datentypen haben DB-Systeme normalerweise eingebaute Interpretationsfunktionen. Zum Beispiel würden die Konstanten $\Scodelit{\quoted{42}}$ und $\Scodelit{\quoted{+42}}$ dieselbe Ganzzahl bezeichnen.

\end{frame}

\begin{frame}\frametitle{Zeitkomplexität}

Sei $m$ die Größe von $F$ und $n=|\Inter|$ (Gesamtgröße der Datenbank)
\bigskip

\begin{itemize}
\item Maximale Rekursionstiefe?\\\pause
$\leadsto$ beschränkt durch Zahl der Teilformeln: $\leq m$
%
\item Maximale Zahl der Iterationen (und rekursiven Aufrufe) in $\Scode{for}$-Schleife?\\\pause
$\leadsto$ $|\Delta^\Inter|\leq n$ pro rekursivem Aufruf\\
$\leadsto$ insgesamt $\leq n^m$ Iterationen\pause
%
\item Rekursive Aufrufe in anderen Fällen?\\
$\leadsto$ 1 oder 2 \pause
%
\item $\tuple{c_1^{\Inter,\Zuweisung},\ldots,c_n^{\Inter,\Zuweisung}}\in p^\Inter$ ist entscheidbar in linearer Zeit bzgl.\ $n$
\end{itemize}
\bigskip\pause

Gesamtlaufzeit in $\max(n,2)^m\cdot n \leq (n+2)^{m+1}$:
\begin{itemize}
\item Komplexität des Algorithmus: \alert{in \ExpTime}
\item Komplexität bzgl. Größe der Datenbank ($m$ konstant): \alert{in \PTime}
\end{itemize}

\end{frame}

\begin{frame}\frametitle{Speicherkomplexität}

Wir erhalten eine bessere Komplexitätsabschätzung, wenn wir den Speicherbedarf
betrachten
\bigskip

Sei $m$ die Größe von $F$ und $n=|\Inter|$ (Gesamtgröße der Datenbank)
\bigskip

\begin{itemize}
\item Speichere pro (rekursivem) Aufruf einen Pointer auf eine Teilformel von $F$: $\log m$
\item Speichere für jede Variable in $F$ (maximal $m$) die aktuelle Zuweisung (als Pointer): $m\cdot\log n$
\item $\tuple{c_1^{\Inter,\Zuweisung},\ldots,c_n^{\Inter,\Zuweisung}}\in p^\Inter$ ist entscheidbar in logarithmischem Speicher bzgl.\ $n$
\end{itemize}
\bigskip\pause

Speicher in \ghost{$m\log m + m\log n + \log n = m\log m + (m+1)\log n$}
\begin{itemize}
\item Komplexität des Algorithmus: \alert{in \PSpace}
\item Komplexität bzgl. Größe der Datenbank ($m$ konstant): \alert{in \LogSpace}
\end{itemize}

{\tiny\emph{Zur Erinnerung:} $\PSpace\subseteq\ExpTime$ und $\LogSpace\subseteq\PTime$, d.h. die obigen Schranken sind besser}

\end{frame}

\begin{frame}\frametitle{Komplexität des Auswertungsproblems}

\theobox{\emph{Satz:} Das Auswertungsproblem der Prädikatenlogik ist $\PSpace$-vollständig.}\pause

\emph{Beweis:} Durch Reduktion vom Auswertungsproblem quantifizierter Boolscher Formeln (\Slang{TrueQBF}).\medskip

Sei $\quantor_1 p_1. \quantor_2 p_2. \cdots \quantor_n p_n.F[p_1,\ldots,p_n]$ eine QBF (mit $\quantor_i\in\{\forall,\exists\}$)\pause
\begin{itemize}
\item Datenbankinstanz $\Inter$ mit $\Delta^\Inter=\{0,1\}$
\item Eine Tabelle mit einer Spalte: $\textsf{true}(1)$
\item Aus der gegebenen QBF erstellen wir die folgende prädikatenlogische Formel ohne freie Variablen:
\[\quantor_1 x_1. \quantor_2 x_2. \cdots \quantor_n x_n.F[p_1/\textsf{true}(x_1),\ldots,p_n/\textsf{true}(x_n)]\]
wobei $F[p_1/\textsf{true}(x_1),\ldots,p_n/\textsf{true}(x_n)]$ die Formel ist, die aus $F$ entsteht, wenn man jedes aussagenlogische Atom $p_i$ durch das prädikatenlogische Atom
$\textsf{true}(x_n)$ ersetzt.
\end{itemize}
Die Korrektheit dieser Reduktion ist leicht zu zeigen. \qed

\end{frame}

\begin{frame}\frametitle{Wie schwer sind Datenbankabfragen?}

\theobox{\emph{Korollar:} Die Beantwortung von SQL-Anfragen ist \PSpace-schwer, sogar wenn die Datenbank nur eine einzige Tabelle mit einer einzigen Zeile enthält.}\bigskip

Die Komplexität steckt vor allem in der Struktur der Anfrage.\bigskip

Ist die Anfrage fest vorgegeben oder in ihrer Größe beschränkt, dann wird das praktische Verhalten oft von der Datenbankgröße dominiert: Bezüglich dieser Größe ist das Problem aber in $\LogSpace$.\bigskip

Man kann sogar noch niedrigere Komplexitätsschranken bzgl. der Datenbankgröße angeben (siehe Vorlesung Database Theory).
\bigskip

\alert{$\leadsto$ SQL-Anfragebeantwortung ist praktisch implementierbar, aber nur solange die Anfragen nicht zu komplex werden.}

\end{frame}

\sectionSlide{Die Grenzen der Prädikatenlogik}

\begin{frame}\frametitle{Reprise: Formeln als Anfragen}

{\scriptsize

\begin{tabular}[t]{|l|l|}
\multicolumn{2}{@{}l}{linien:}\\
\hline
\textbf{Linie} & \textbf{Typ} \\
\hline
85 & Bus \\\hline
3 & Tram \\\hline
F1 & Fähre \\\hline
\ldots & \ldots\\\hline
\end{tabular}
\hspace{15mm}
\begin{tabular}[t]{|l|l|l|}
\multicolumn{2}{@{}l}{haltestellen:}\\
\hline
\textbf{SID} & \textbf{Name} & \textbf{Rollstuhl}\\
\hline
17 & Hauptbahnhof & true\\\hline
42 & Helmholtzstr. & true\\\hline
57 & Stadtgutstr. & true\\\hline
123 & Gustav-Freytag-Str. & false\\\hline
\ldots & \ldots & \ldots\\\hline
\end{tabular}

\begin{tabular}[t]{|l|l|l|}
\multicolumn{2}{@{}l}{verbindung:}\\
\hline
\textbf{Von} & \textbf{Zu} & \textbf{Linie}\\
\hline
57  & 42 & 85 \\\hline
17  & 789 & 3 \\\hline
\ldots & \ldots & \ldots\\\hline
\end{tabular}
\hspace{0.5cm}
% 
% \medskip
\ghost{
\begin{minipage}[t]{9cm}
\vspace{0.0cm}
Die einfache Arität der Prädikatenlogik wird durch ein\\ \redalert{Schema} mit Namen (und oft auch Datentypen) ersetzt:
\begin{itemize}
\item linien[Linie:string, Typ:string]
\item haltestellen[SID:int, Halt:string, Rollstuhl:bool]
\item verbindung[Von:int, Zu:int, Linie:string]
\end{itemize}
Relationale Algebra: Parameter (Spalten) durch Namen adressiert\\
Prädikatenlogik: Parameter durch Reihenfolge adressiert
\end{minipage}}}\vspace{1cm}

Die Anfrage \alert{$\exists z_{\text{Linie}}.(\textsf{verbindung}(x_\text{Von},x_\text{Zu}, z_{\text{Linie}})\wedge \textsf{linien}(z_{\text{Linie}},x_{\text{Typ}}))$}
entspricht einer (natürlichen) Join-Operation ($\wedge$) mit anschließender Projektion ($\exists$):
\alert{$\pi_{\text{Von},\text{Zu},\text{Typ}}(\textsf{verbindung}\bowtie\textsf{linien})$}.

\end{frame}

\begin{frame}\frametitle{Prädikatenlogik als Anfragesprache}

\emph{Beispielanfragen:}\medskip

\alert{``Haltestellen, die Helmholtzstr. sind:''}
%
\[ Q_0[x_0] = (x_0\approx 42) \]\pause
% \[R_0 = \{\{\text{From}\mapsto 42\}\big\}\]\pause
%
\alert{``Haltestellen direkt neben Helmholtzstr.:''}
\[
   Q_1[x_1] = \exists x_0,z_\text{Linie}.(\text{verbindung}(x_0,x_1,z_\text{Linie})\wedge Q_0[x_0])
\]\pause
% \[
% R_1 = \delta_{\text{To}\to\text{From}}(\pi_{\text{To}}(\text{Connect}\bowtie R_0))
% \]\pause
\alert{``Haltestellen, die zwei Halte weit von Helmholtzstr. entfernt sind:''}
\[
   Q_2[x_2] = \exists x_1,z_\text{Linie}.(\text{verbindung}(x_1,x_2,z_\text{Linie})\wedge Q_1[x_1])
\]\pause
Und so weiter \ldots\bigskip
% \[
% R_2 = \delta_{\text{To}\to\text{From}}(\pi_{\text{To}}(\text{Connect}\bowtie R_1))
% \]\pause
% 

\alert{``Haltestellen, die von Helmholtzstr. aus mit einem Kurzstreckenticket erreichbar sind:''}
\[
	Q_0[x]\vee Q_1[x]\vee Q_2[x]\vee Q_3[x]\vee Q_4[x]
\]
% \[
% R_0\cup R_1\cup R_2 \cup R_3 \cup R_4
% \]
% 
% What about all stops reachable from Helmholtzstr.?\\\pause
% $\leadsto$ see upcoming lectures \ldots
% 
\end{frame}


\begin{frame}\frametitle{Die Grenzen der Prädikatenlogik}

\redalert{Wie finden wir alle Haltestellen, die man von Helmholtzstr. aus erreichen kann?}\pause\medskip

Es stellt sich heraus, dass das unmöglich ist.\bigskip

\anybox{strongyellow}{\emph{Intuition:}
Prädikatenlogik kann nur ``lokale'' Eigenschaften überprüfen.
}\medskip\pause

Das kann man mathematisch ausdrücken:\medskip

\anybox{strongyellow}{\emph{Vage Behauptung (frei nach Gaifman's Locality Theorem):}
Für jeden Satz $F$ gibt es eine Zahl $d$, so dass für beliebige Interpretationen $\Inter$ und $\Jnter$ gilt: 
\begin{itemize}
\item Wenn man nur zusammenhängende endliche Teile von $\Inter$ und $\Jnter$ betrachtet, die höchstens Pfade der Länge $d$ enthalten (``Durchmesser $\leq d$'')
\item und wenn sich $\Inter$ und $\Jnter$ bezüglich dieser $d$-Umgebungen nicht unterscheiden,
\item dann kann auch $F$ die Interpretationen nicht unterscheiden: $\Inter\models F$ gdw. $\Jnter\models F$.
\end{itemize}
}

\textcolor{devilscss}{Der Durchmesser $d$, der angibt wie weit $F$ höchstens schauen kann, hängt exponentiell von der Schachtelungstiefe der Quantoren ab.}

\end{frame}

\begin{frame}\frametitle{Rekursive Anfragen}

Nichtlokale Eigenschaften wie die Erreichbarkeit in Graphen sind praktisch relevant (speziell in Graphdatenbanken)\medskip

\redalert{Wie kann man solche Anfragen logisch ausdrücken?}\bigskip\pause

\emph{Idee:} Um beliebig weit zu schauen muss man Rekursion einführen.\bigskip

\examplebox{\emph{Beispiel:} Eine Haltestelle ist von Helmholtzstr. aus erreichbar, wenn
\begin{enumerate}[(1)]
\item sie die Haltestelle Helmholtzstr. ist, oder
\item sie neben einer Haltestelle liegt, die von Helmholtzstr. aus erreichbar ist.
\end{enumerate}}

\end{frame}

\begin{frame}\frametitle{Zusammenfassung und Ausblick}

Beschränkt man Prädikatenlogik auf endliche Modelle, so gibt es kein vollständiges und korrektes Verfahren zum logischen Schließen -- dafür wird Erfüllbarkeit semi-entscheidbar\bigskip

Auswertungsproblem auf endlichen Modellen = Anfragebeantwortung in Datenbanken\\
(\PSpace-vollständig, aber sub-polynomiell bzgl. Datenbankgröße)\bigskip

Prädikatenlogik hat Grenzen: 
\begin{itemize}
\item bei der Modellierung (logisches Schließen), z.B. transitiver Abschluss
\item bei logischen Abfragen (Model Checking), z.B. Erreichbarkeit
\end{itemize}\bigskip

\anybox{yellow}{
Was erwartet uns als Nächstes?
\begin{itemize}
\item Datalog und Logik höherer Ordnung
\item Gödel
\item Probeklausur und 2. Repetitorium
\end{itemize}
}

\end{frame}

% \begin{frame}\frametitle{Zusammenfassung und Ausblick}
% 
% Die prädikatenlogische Resolution ist ein vollständiges und korrektes Verfahren für die Unerfüllbarkeit logischer Formeln\bigskip
% 
% In gewissem Sinne ist Prädikatenlogik eine Kurzschreibweise für möglicherweise unendliche aussagenlogische Theorien
% \bigskip
% 
% Bei Beschränkung auf endliche Modelle gibt es kein vollständiges und korrektes Verfahren zum logischen Schließen -- dafür wird Erfüllbarkeit semi-entscheidbar\bigskip
% 
% Auswertungsproblem auf endlichen Modellen = Anfragebeantwortung in Datenbanken
% 
% \anybox{yellow}{
% Was erwartet uns als nächstes?
% \begin{itemize}
% \item Komplexität des Auswertungsproblems
% \item Gödel
% \item Probeklausur und 2. Repetitorium
% \end{itemize}
% }
% 
% \end{frame}


% \begin{frame}[t]\frametitle{Bildrechte}
% 
% Folie \ref{frame_herbrand}: Fotografie von Natasha Artin Brunswick, 1931, CC-By 3.0
% 
% \end{frame}


\end{document}
